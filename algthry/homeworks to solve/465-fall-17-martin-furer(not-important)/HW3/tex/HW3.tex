\documentclass[12pt]{article}

\usepackage{color}
\usepackage{float}
\usepackage{answers}
\usepackage{setspace}
\usepackage{graphicx}
\usepackage{enumitem}
\usepackage{multicol}
\usepackage{mathrsfs}
\usepackage[margin=1in]{geometry} 
\usepackage{amsmath,amsthm,amssymb}
 
\newcommand{\N}{\mathbb{N}}
\newcommand{\Z}{\mathbb{Z}}
\newcommand{\C}{\mathbb{C}}
\newcommand{\R}{\mathbb{R}}

\DeclareMathOperator{\sech}{sech}
\DeclareMathOperator{\csch}{csch}
 
\newenvironment{theorem}[2][Theorem]{\begin{trivlist}
\item[\hskip \labelsep {\bfseries #1}\hskip \labelsep {\bfseries #2.}]}{\end{trivlist}}
\newenvironment{definition}[2][Definition]{\begin{trivlist}
\item[\hskip \labelsep {\bfseries #1}\hskip \labelsep {\bfseries #2.}]}{\end{trivlist}}
\newenvironment{proposition}[2][Proposition]{\begin{trivlist}
\item[\hskip \labelsep {\bfseries #1}\hskip \labelsep {\bfseries #2.}]}{\end{trivlist}}
\newenvironment{lemma}[2][Lemma]{\begin{trivlist}
\item[\hskip \labelsep {\bfseries #1}\hskip \labelsep {\bfseries #2.}]}{\end{trivlist}}
\newenvironment{exercise}[2][Exercise]{\begin{trivlist}
\item[\hskip \labelsep {\bfseries #1}\hskip \labelsep {\bfseries #2.}]}{\end{trivlist}}
\newenvironment{solution}[2][Solution]{\begin{trivlist}
\item[\hskip \labelsep {\bfseries #1}]}{\end{trivlist}}
\newenvironment{problem}[2][Problem]{\begin{trivlist}
\item[\hskip \labelsep {\bfseries #1}\hskip \labelsep {\bfseries #2.}]}{\end{trivlist}}
\newenvironment{question}[2][Question]{\begin{trivlist}
\item[\hskip \labelsep {\bfseries #1}\hskip \labelsep {\bfseries #2.}]}{\end{trivlist}}
\newenvironment{corollary}[2][Corollary]{\begin{trivlist}
\item[\hskip \labelsep {\bfseries #1}\hskip \labelsep {\bfseries #2.}]}{\end{trivlist}}
 
\begin{document}
% --------------------------------------------------------------
%                         Start here
% --------------------------------------------------------------
 
\title{\textbf{HW3}}%replace with the appropriate homework number
\author{Seyed Armin Vakil Ghahani\\ %replace with your name
PSU ID: 914017982\\
CSE-565 Fall 2018\\
Collaboration with:
Sara Mahdizadeh Shahri, Soheil Khadirsharbiyani\\
Muhammad Talha Imran} %if necessary, replace with your course title}
 
\maketitle
%Below is an example of the problem environment
\begin{problem}{1}
Sum of Functions
\end{problem}

%Below is the solution environment
\begin{solution}{}
\begin{itemize}
\item a) For each i, $f_i(n) \in O(n)$. Hence, there is an $N_i$ and $c_i$ that inequality
$f_{i} (n) \leq c_{i}*n$ is true for each $n \geq N_i$.
Thus, for each $n \geq max_{n=1}^{m} N_i$:
$$s_m(n) \leq \sum_{i=1}^{m} c_i*n \Rightarrow s_m(n) \leq n*C$$
which $C = \sum_{i=1}^{m} c_i$. Therefore, it proves that $s_m(n) \in O(n)$.

\item b) For each i, $f_i(n) \in O(n)$. Hence, there is an $N_i$ and $c_i$ that inequality
$f_{i} (n) \leq c_{i}*n$ is true for each $n \geq N_i$.
Thus, for each $n \geq max_{i=1}^{n} N_i$:
$$s_n(n) \leq \sum_{i=1}^{n} c_i*n \Rightarrow s_n(n) \leq n*c_k*\sum_{i=1}^{n} \frac{c_i}{c_k}$$
which $c_k = max_{i=1}^{n} c_i$. Hence, for each $1 \leq i \leq n$, $c_i \leq c_k \Rightarrow
\frac{c_i}{c_k} \leq 1$. So,
$$s_n(n) \leq n * c_k * \sum_{i=1}^{n} 1 = n*c_k*n = n^2*c_k$$
Therefore, $s_n(n) \in O(n^2)$.

\end{itemize}
\end{solution}

\begin{problem}{2}
Shortest Cycles
\end{problem}

%Below is the solution environment
\begin{solution}{}
\begin{itemize}
\item a) We are going to prove that the vertexes $v_i, ..., v_n, v_1, ..., v_{i-1}$ are on depth
$0$ through $k-1$ consecutively. It will be proven by contradiction. Suppose that, the first
vertex that breaks this rule is $v_{j+1}$ that is after $v_j$, and $v_j$ and $v_{j+1}$ are in 
depth $a \geq b$, consecutively. The length of path from $v_i$ to $v_{j+1}$ is $b$ through
the BFS tree, however, the length of this path through the cycle is $a+1$ which is greater
than $b$ (not equal). Hence, if we replace the path in the BFS tree with the path in the cycle,
the length of the shortest cycle will be reduced that is in contradiction with the shortest cycle
assumption.

\item b) If we run a BFS algorithm from a node in the graph, it will give us the shortest cycle
that concludes this vertex. Whenever a back edge to the root is shown in the BFS algorithm,
it represents a cycle which includes root. Since, in BFS, the vertexes are traversed by their
distance to the root, the first back edge to the root will be the shortest cycle. Hence, if we run
this algorithm for every vertex in the graph, the shortest cycle of each vertex will be calculated
and we can take the shortest cycle of the graph in O(n*(m+n)) because the BFS algorithm is O(m+n)
and we run it for n times. 

\end{itemize}
\end{solution}



\pagebreak

\end{document}

